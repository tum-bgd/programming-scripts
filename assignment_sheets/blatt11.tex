\documentclass[twoside]{article}
\usepackage[utf8]{inputenc}
%\usepackage[uebung, answers]{tumbgdm}
\usepackage[uebung]{tumbgdm}
\usepackage{lecturesetupcf2}
\nummerUebungsblatt{5}
\nummerErsteFrage{11}

\LearningOutcome{
Shortest Paths
}

\begin{document}
\maketitle

\begin{task}{Dijkstra}{}{}
  Run the Dijkstra Algorithm on Paper using the example from Slide 35
\end{task}

\begin{task}{C++-Dijkstra and Dot}{}{}
  Compile (use linux, might be easier, Docker or VirtualBox can help if you are on Windows or Mac).
  Compare your solution to the output of this program.
\end{task}

\begin{task}{A Baseline Demo}{}{}
  The GISCUP 2015 implementation at \url{https://github.com/mwernerds/giscup2015} contains a few additional shortest path algorithms. Please ignore them, they are
  not obvious. However, the whole framework is carefully organized, try to understand it from a software technology
  point of view.

  There is a benchmark part which has no GUI and no useless debugging code to measure speed. There is a GUI which provides visualization, scrolling, and interaction,

  There is a demo mode in the GUI, where random paths are computed. 

  All of this is based on Boost and wxWidgets, which are complicated, but powerful libraries as they can bridge
  Linux, Windows and Mac with ease.

  Try compiling and see if you succeed. Let us know your challenges. Advanced students should try to cross-compile on Linux for Windows x64 using the MXE M Cross Environment.
  
  OpenGL (plain old OpenGL) is used to visualize things. This can be handy in your future work...
\end{task}
\end{document}
