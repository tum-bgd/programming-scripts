\begin{task}{Arduino}{}{}
  The Arduino family consists of a set of well-tested PCB boards holding typically AVR microcontrollers that
  have always been quite efficient in terms of simplicity of external circuitry as they require not much more than a clock source and a suitable energy management.

  The Arduino project combines such basic boards with a well-known geometry for extensions (e.g., where are the PINs), a small IDE for the Atmel-AVR-GCC compilers, and a basic firmware that translates operation of such microcontrollers into two functions: setup() and loop().

  In this way, a typical Arduino program called sketch does not contain a main, but rather a setup function which is called once at the beginning and a loop function which is called endlessly.

  If you own an Arduino, this task is easy, if not, I would recommend buying a cheap set (starter kit, sensor kit)
  for learning. However, you are free to answer the questions after an Internet research as well in ``theory''.

  A very simple program looks like
  \begin{lstlisting}
  int buttonPin = 3;

// setup initializes serial and the button pin
void setup() {
  Serial.begin(9600);
  pinMode(buttonPin, INPUT);
}

// loop checks the button pin each time,
// and will send serial if it is pressed
void loop() {
  if (digitalRead(buttonPin) == HIGH) {
    Serial.write('H');
  }
  else {
    Serial.write('L');
  }

  delay(1000);
}
\end{lstlisting}
  
  \begin{itemize}
  \item{Design and implement a circuit (maybe buy an Arduino Nano and a few LEDs for this) with a blinking LED}
  \item{Design and implement the traffic light state machine with LEDs}
  \item{Use PWM using `analogWrite` to dim an LED based on a potentiometer read from an A/D pin}
    
    \end{itemize}

\end{task}
