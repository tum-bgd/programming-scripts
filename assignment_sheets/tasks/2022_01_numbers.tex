\begin{task}{Number Representation Systems}{}{}

  Number Representation in terms of positional number systems is a key insight enabling the representation of more or less
  everything in binary representation. From a digital electronics perspective, this is great as common problems of real-world electronic systems, especially noise, can be removed (mostly!) by having large regions of voltages representing the bits 0 and 1. Furthermore, it is interesting to know that modern (future) hardware might not be as binary anymore. Maybe read on Resistive Random Access Memory (RRAM) if interested.

    
  \begin{enumerate}
  \item{\textbf{Number Conversions:}Convert decimal numbers 10, 4, 2.4, 1.25, and $1 \frac{1}{3}$ into binary, octal, and hexadecimal values.}
  \item{\textbf{Base32} Write a C/C++ program that converts a memory area (given by a pointer and a length variable) into a stream of Base32 characters. Choose one of the available Base32 character sets.}
  \item{\textbf{Conversion Tool}: Write a C/C++ program to convert integer numbers between decimal and binary system.}
  \item{\textbf{FP Conversion*}: Write a program that convers a decimal floating point number given as a string into IEEE 754 float representation.}
    \item{\textbf{GeoHash:} Base32 has been used to define a representation of spatial locations on Earth surface in short strings of characters. Try to find information on it on the Internet and implement a conversion from WGS84 GPS coordinates into strings.}
  \end{enumerate}
%

  
  
\end{task}
